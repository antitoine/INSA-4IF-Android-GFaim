\section{Spécifications Minimales}

    Les spécifications données par le sujet sont les suivantes : \\
    
\begin{itemize}
    \item[\textbullet] Etre disponible sur smartphone.
    \item[\textbullet] Suivre le guide de style d'Android et la charte graphique de l'INSA, dans la mesure du possible.
    \item[\textbullet] Prendre en compte différents types d'utilisateurs et différents types de points de restauration.
    \item[\textbullet] S'adapter aux différents types d'utilisateurs.
    \item[\textbullet] Fournir à l'utilisateur des informations pertinentes dans son contexte d'utilisation. \\
\end{itemize}

    Ces spécifications générales seront étendues dans la suite de cette section. \\
    
    Les spécifications de notre application, concernant les deux premiers items de la liste ci dessus sont décrites dans le tableau ci-après. Elle décrive les différentes contraintes ergonomiques et d'accessibilité qui seront respectées par l'application développée. 

\begin{table}[H]
    \centering
    \caption{Tableau des spécifications minimales}
    \label{min-spec-table}
    \begin{tabular}{p{8cm}|p{8cm}}
        \bf{Spécification} & \bf{Justification} \\ \hline
        Chaque vue présentera une information spécifique & Ne pas diluer l'information importante dans une grande quantité d'information inutile \\ \hline
        Chaque vue présentera un nombre réduit de contrôles ($<= 10$) & \'Eviter d'introduire une complexité d'usage de l'application \\ \hline
        Les contrôles seront regroupés dans les zones accessibles de l'écran &  Faciliter l'accès aux contrôles \\ \hline
        Les contrôles dont la fonctionnalité est la même (ex: suppression) sur des vues différentes partageront la même symbolique & Uniformiser, homogénéiser, la symbolique de l'application \\ \hline
        Les contrôles et zones non utilisables seront grisées & Suggéger à l'utilisateur d'ignorer certaines zones ou contrôles et éviter les erreurs \\ \hline
        Les chaînes de caractères seront affichées avec une police adaptée au media utilisé & Permettre à l'utilisateur de ne pas se fatiguer \\ \hline
        Les polices de caractères et couleurs seront personnalisables dans le menu "Personnalisation" & Permettre à l'utilisateur de paramétrer l'application selon ses besoins \\ \hline
        Les miniatures seront agrandies quand l'utilisateur clique dessus & Permettre à l'utilisateur de voir l'image quelque soit sa condition \\ \hline
        L'information principale sera présentée au centre de la vue & Faciliter l'accès à l'information en respectant les normes d'ergonomie \\ \hline
        Un bandeau présentant les contrôles globaux sera présent sur chaque vue & Donner accès aux contrôles depuis n'importe où dans l'application \\ \hline
        Un menu masquable sera présent sur la gauche de l'application et donnera accès aux principales fonctionnalités de l'application & Permettre à l'utilisateur de naviguer efficacement \\ \hline
        La profondeur de ce menu ne devra pas excéder deux niveaux & Conserver une interface simple et lisible pour l'utilisateur \\ \hline 
        Les messages d'erreur contiendront une description brève de l'erreur et la démarche à suivre pour la résoudre & Guider l'utilisateur de manière efficace en cas d'erreur \\ \hline
        Les notifications seront réglables pour laisser la liberté à l'utilisateur de les activer ou non & Ne pas imposer une fonctionnalité à l'utilisateur \\ \hline
        Les enchainement de vue de type processus seront limités à trois étapes si ils existent & Ne pas lasser l'utilisateur avec une procédure trop longue \\ \hline
        Les gestes et les inputs tactiles seront utilisés pour les barres de défilement qui n'apparaitront pas sur l'interface & Alléger l'interface graphique et améliorer l'expérience utilisateur en respectant les principes intuitifs \\
    \end{tabular}
\end{table}

    Le schéma ci-après décrit l'enchaînement des vues de l'application.
    
    \todo{Insérer le schéma d'enchainement des fenêtres ici}