\section{Spécifications Avancées}

3a. L'application prend en compte différents types d'utilisateurs. \\
Il y a deux profils principaux : le profil consommateur, qui inclue les étudiants du campus, les professeurs, les employés, etc., et le profil restaurateur, identique au profil consommateur à cela près qu'il peut administrer la page de son restaurant. \\
Fonctionnalités principales du profil consommateur : accéder à la liste des restaurants à proximité et consulter leur détail (photos, menu, distance, temps d'attente, avis d'autres consommateurs...), voir les restaurants alentours sur une carte, régler des alarmes à une heure donnée pour le prévenir de l'état des indicateurs de certains restaurants. \\
Fonctionnalités principales du profil restaurateur : les mêmes que celles du profil consommateur, avec une fonctionnalité principale supplémentaire : administrer la page de son propre restaurant.\\

3b. L'application prend en compte différents types de restaurants.\\
Lors de la recherche et de l'affichage de restaurants, il est possible de les trier selon leur type : restaurant, restaurant universitaire, restaurant-bar...

4. L'application s'adapte aux différents types d'utilisateurs.\\

Consommateur : \\
\begin{table}[H]
    \centering
    \caption{Tableau des spécifications avancées du consommateur}
    \label{min-spec-table}
    \begin{tabular}{p{8cm}|p{8cm}}
 	\bf{Spécification} & \bf{Justification} \\ \hline
		Le consommateur peut accéder rapidement à tous les restaurants alentours grâce au menu, section "Restaurants". & L'intérêt principal de l'application pour le consommateur est de consulter les restaurants à proximité. \\ \hline
		Le consommateur peut rechercher un restaurant, soit directement avec son nom s'il le connaît, soit en indiquant des critères de recherche (temps d'attente, distance, etc.) pour une recherche avancée. & Cela permet au consommateur d'accéder rapidement à un/des restaurant(s) selon ce qu'il souhaite : obtenir l'état d'un restaurant précis, ou bien trouver un restaurant où il pourrait manger selon des critères qu'il aura définis. \\ \hline
		Le consommateur peut consulter les détails du restaurant, ce qui permet d'avoir une vue plus détaillée d'un restaurant choisi, notamment avec les avis d'autres utilisateurs et une mini-vue carte. & Ceci apporte un accès rapide aux indicateurs du restaurant, tout en gardant une vue compacte (certaines autres fonctionnalités étant accessible depuis cette vue). \\ \hline
		Depuis la vue détaillée du restaurant, le consommateur peut accéder à certaines autres fonctionnalités : photos du restaurant, téléphone, lien vers son site internet, laisser un avis, ajouter aux favoris, voir le menu. & Mettre toutes ces fonctionnalités directement dans la vue détaillée du restaurant aurait conduit à une page très remplie et peu lisible, d'où le choix de proposer d'y accéder seulement si le consommateur le souhaite. \\ \hline
		Le consommateur peut se repérer et voir les restaurants alentours à tout instant grâce au menu, section "Carte". & Le consommateur peut vouloir repérer un restaurant à proximité d'un endroit où il souhaite aller avant ou après manger. \\ \hline
		Le consommateur peut programmer une alarme pour être prévenu de l'état des indicateurs de certains restaurants, grâce à une notification, et donc sans avoir à ouvrir l'application. & La rapidité de l'accès à l'information est un critère primordial pour ce type d'application. \\
    \end{tabular}
\end{table}

Restaurateur : \\

\begin{table}[H]
    \centering
    \caption{Tableau des spécifications avancées du restaurateur}
    \label{min-spec-table}
    \begin{tabular}{p{8cm}|p{8cm}}
        \bf{Spécification} & \bf{Justification} \\ \hline
Le restaurateur peut changer les informations générales de son restaurant : numéro de téléphone, adresse... & Il est utile que le consommateur soit informé rapidement de tels changements. \\ \hline
Le restaurateur peut ajouter des photos de son propre restaurant. & Il est intéressant pour le consommateur de voir à quoi ressemble le restaurant pour s'en faire une idée. \\ \hline
Le restaurateur peut ajouter le menu de son restaurant. & Le consommateur peut vouloir consulter ce qu'il y a à manger au restaurant sans avoir à y aller. \\ \hline
Le restaurateur n'a pas besoin de renseigner le temps d'attente dans son restaurant : cela est fait en temps réel grâce à un système annexe. & Entrer les temps d'attente à la main pourrait être faussé, incomplet et peu précis. \\
    \end{tabular}
\end{table}