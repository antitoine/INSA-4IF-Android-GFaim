\input{preambule}

%%  ========   IMPORTANT ========
%% Indiquer ici les parties que vous voulez compilez

%%% Begin document
\begin{document}
\includepdf[pages={1}]{title.pdf}

\listoftodos[Todo List]
\newpage
\tableofcontents
{%
\let\oldnumberline\numberline%
\renewcommand{\numberline}{\figurename~\oldnumberline}%
\renewcommand\listfigurename{Liste des figures}
\listoffigures%
}
{%
\let\oldnumberline\numberline%
\renewcommand{\numberline}{\tablename~\oldnumberline}%
\renewcommand\listtablename{Liste des tableaux}
\listoftables%
}

\newpage
\addcontentsline{toc}{part}{Introduction}
\chapter*{Introduction}
\section*{Présentation du contenu du rapport}

\appname n'est pas qu'une simple application permettant à l'utilisateur de choisir un restaurant, c'est bien plus que cela. \appname, c'est la garantie de ne jamais sortir d'un restaurant le ventre vide, et de ne jamais y entrer sans savoir si l'attente sera plus longue que le temps dont nous disposons. En effet, qui n'est jamais sorti déçu d'un restaurant, sans avoir pu manger car le temps d'attente était bien trop long, ou bien les prix trop élevés ? Grâce à \appname, ce ne sera plus un souci !\\
Le but de cette application est d'offrir à l'utilisateur un moyen simple et intuitif de repérer tous les restaurants alentours, leur distance à pied, et même l'attente en temps réel. Mais de nombreuses autres fonctionnalités existent également : programmer des notifications pour être prévenu de l'affluence de son restaurant favori juste avant d'aller manger, visualiser tous les restaurants proches sur une carte, ou encore voir les avis d'autres utilisateurs sur tel ou tel point de restauration.\\
En réalisant cette application, notre objectif était simple : répondre au besoin grandissant d'utilisateurs souhaitant optimiser leur emploi du temps. En effet, dans un environnement où tout évolue rapidement et où le temps se fait précieux, il est impensable de devoir attendre une heure avant de pouvoir être servi. Cependant, dans le cadre de ce projet, nous nous arrêtons à l'aspect visuel de l'application, en ne réalisant que les interfaces.

\part{Analyse des besoins}
\setcounter{section}{0}
\todo{inclure ici le fichier d'analyse des besoins}

\part{Analyse de l'existant}
\setcounter{section}{0}
\todo{inclure ici le fichier d'analyse de l'existant}

\part{Maquettes}
\setcounter{section}{0}
\todo{inclure ici le fichier contenant les maquettes}

\part{Spécifications}
\setcounter{section}{0}
\section{Spécifications Minimales}

    Les spécifications données par le sujet sont les suivantes : \\
    
\begin{itemize}
    \item[\textbullet] Être disponible sur smartphone.
    \item[\textbullet] Suivre le guide de style d'Android et la charte graphique de l'INSA, dans la mesure du possible.
    \item[\textbullet] Prendre en compte différents types d'utilisateurs et différents types de points de restauration.
    \item[\textbullet] S'adapter aux différents types d'utilisateurs.
    \item[\textbullet] Fournir à l'utilisateur des informations pertinentes dans son contexte d'utilisation. \\
\end{itemize}

    Ces spécifications générales seront étendues dans par la suite. \\
    
    Les spécifications de notre application, concernant les deux premiers items de la liste ci dessus sont décrites dans le tableau ci-après. Elle décrive les différentes contraintes ergonomiques et d'accessibilité qui seront respectées par l'application développée. 

\begin{table}[H]
    \centering
    \caption{Tableau des spécifications minimales}
    \label{min-spec-table}
    \begin{tabular}{p{8cm}|p{8cm}}
        \bf{Spécification} & \bf{Justification} \\ \hline
        Chaque vue présentera une information spécifique & Ne pas diluer l'information importante dans une grande quantité d'informations inutiles \\ \hline
        Chaque vue présentera un nombre réduit de contrôles ($<= 10$) & \'Eviter d'introduire une complexité d'usage dans l'application \\ \hline
        Les contrôles seront regroupés dans les zones accessibles de l'écran &  Faciliter l'accès aux contrôles \\ \hline
        Les contrôles dont la fonctionnalité est la même (ex: suppression) sur des vues différentes partageront la même symbolique & Uniformiser, homogénéiser la symbolique de l'application \\ \hline
        Les contrôles et zones non utilisables seront grisées & Suggéger à l'utilisateur d'ignorer certaines zones ou contrôles et éviter les erreurs \\ \hline
        Les chaînes de caractères seront affichées avec une police adaptée au media utilisé & Permettre à l'utilisateur de ne pas se fatiguer \\ \hline
        Les polices de caractères et couleurs seront personnalisables dans le menu \og{}Personnalisation\fg{} & Permettre à l'utilisateur de paramétrer l'application selon ses besoins \\ \hline
        Les miniatures seront agrandies quand l'utilisateur clique dessus & Permettre à l'utilisateur de voir l'image quelque soit sa condition \\ \hline
        L'information principale sera présentée au centre de la vue & Faciliter l'accès à l'information en respectant les normes d'ergonomie \\ \hline
        Un bandeau présentant les contrôles globaux sera présent sur chaque vue & Donner accès aux contrôles depuis n'importe où dans l'application \\ \hline
        Un menu masquable sera présent sur la gauche de l'application et donnera accès aux principales fonctionnalités de l'application & Permettre à l'utilisateur de naviguer efficacement \\ \hline
        La profondeur de ce menu ne devra pas excéder deux niveaux & Conserver une interface simple et lisible pour l'utilisateur \\ \hline 
        Les messages d'erreur contiendront une description brève de l'erreur et la démarche à suivre pour la résoudre & Guider l'utilisateur de manière efficace en cas d'erreur \\ \hline
        Les notifications seront réglables pour laisser la liberté à l'utilisateur de les activer ou non & Ne pas imposer une fonctionnalité à l'utilisateur \\ \hline
        Les enchainements de vue de type processus seront limités à trois étapes si ils existent & Ne pas lasser l'utilisateur avec une procédure trop longue \\ \hline
        Les gestes et les inputs tactiles seront utilisés pour les barres de défilement qui n'apparaitront pas sur l'interface & Alléger l'interface graphique et améliorer l'expérience utilisateur en respectant les principes intuitifs \\
    \end{tabular}
\end{table}

    Le schéma ci-après décrit l'enchaînement des vues de l'application.
    
    \todo{Insérer le schéma d'enchainement des fenêtres ici}
\section{Spécifications Avancées}

3a. L'application prend en compte différents types d'utilisateurs. \\
Il y a deux profils principaux : le profil consommateur, qui inclue les étudiants du campus, les professeurs, les employés, etc., et le profil restaurateur, identique au profil consommateur à cela près qu'il peut administrer la page de son restaurant. \\
Fonctionnalités principales du profil consommateur : accéder à la liste des restaurants à proximité et consulter leur détail (photos, menu, distance, temps d'attente, avis d'autres consommateurs...), voir les restaurants alentours sur une carte, régler des alarmes à une heure donnée pour le prévenir de l'état des indicateurs de certains restaurants. \\
Fonctionnalités principales du profil restaurateur : les mêmes que celles du profil consommateur, avec une fonctionnalité principale supplémentaire : administrer la page de son propre restaurant.\\

3b. L'application prend en compte différents types de restaurants.\\
Lors de la recherche et de l'affichage de restaurants, il est possible de les trier selon leur type : restaurant, restaurant universitaire, restaurant-bar...

4. L'application s'adapte aux différents types d'utilisateurs.\\

Consommateur : \\
\begin{table}[H]
    \centering
    \caption{Tableau des spécifications avancées du consommateur}
    \label{min-spec-table}
    \begin{tabular}{p{8cm}|p{8cm}}
 	\bf{Spécification} & \bf{Justification} \\ \hline
		Le consommateur peut accéder rapidement à tous les restaurants alentours grâce au menu, section "Restaurants". & L'intérêt principal de l'application pour le consommateur est de consulter les restaurants à proximité. \\ \hline
		Le consommateur peut rechercher un restaurant, soit directement avec son nom s'il le connaît, soit en indiquant des critères de recherche (temps d'attente, distance, etc.) pour une recherche avancée. & Cela permet au consommateur d'accéder rapidement à un/des restaurant(s) selon ce qu'il souhaite : obtenir l'état d'un restaurant précis, ou bien trouver un restaurant où il pourrait manger selon des critères qu'il aura définis. \\ \hline
		Le consommateur peut consulter les détails du restaurant, ce qui permet d'avoir une vue plus détaillée d'un restaurant choisi, notamment avec les avis d'autres utilisateurs et une mini-vue carte. & Ceci apporte un accès rapide aux indicateurs du restaurant, tout en gardant une vue compacte (certaines autres fonctionnalités étant accessible depuis cette vue). \\ \hline
		Depuis la vue détaillée du restaurant, le consommateur peut accéder à certaines autres fonctionnalités : photos du restaurant, téléphone, lien vers son site internet, laisser un avis, ajouter aux favoris, voir le menu. & Mettre toutes ces fonctionnalités directement dans la vue détaillée du restaurant aurait conduit à une page très remplie et peu lisible, d'où le choix de proposer d'y accéder seulement si le consommateur le souhaite. \\ \hline
		Le consommateur peut se repérer et voir les restaurants alentours à tout instant grâce au menu, section "Carte". & Le consommateur peut vouloir repérer un restaurant à proximité d'un endroit où il souhaite aller avant ou après manger. \\ \hline
		Le consommateur peut programmer une alarme pour être prévenu de l'état des indicateurs de certains restaurants, grâce à une notification, et donc sans avoir à ouvrir l'application. & La rapidité de l'accès à l'information est un critère primordial pour ce type d'application. \\
    \end{tabular}
\end{table}

Restaurateur : \\

\begin{table}[H]
    \centering
    \caption{Tableau des spécifications avancées du restaurateur}
    \label{min-spec-table}
    \begin{tabular}{p{8cm}|p{8cm}}
        \bf{Spécification} & \bf{Justification} \\ \hline
Le restaurateur peut changer les informations générales de son restaurant : numéro de téléphone, adresse... & Il est utile que le consommateur soit informé rapidement de tels changements. \\ \hline
Le restaurateur peut ajouter des photos de son propre restaurant. & Il est intéressant pour le consommateur de voir à quoi ressemble le restaurant pour s'en faire une idée. \\ \hline
Le restaurateur peut ajouter le menu de son restaurant. & Le consommateur peut vouloir consulter ce qu'il y a à manger au restaurant sans avoir à y aller. \\ \hline
Le restaurateur n'a pas besoin de renseigner le temps d'attente dans son restaurant : cela est fait en temps réel grâce à un système annexe. & Entrer les temps d'attente à la main pourrait être faussé, incomplet et peu précis. \\
    \end{tabular}
\end{table}

\part{Documentation}
\setcounter{section}{0}
\todo{inclure ici le fichier contenant la documentation de l'application}

\part{Bilan technique}
\setcounter{section}{0}
\todo{inclure ici le fichier du bilan technique}

\part{Suivi de projet}
\setcounter{section}{0}
\todo{inclure ici le fichier du suivi de projet}

%%% End document
\end{document}
